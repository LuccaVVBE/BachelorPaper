%---------- Inleiding ---------------------------------------------------------

\section{Introductie}%
\label{sec:introductie}
\subsection{Energie-efficiëntie in software ontwikkeling}
Momenteel staat de ICT-sector gekend als een grote elektriciteitsverbruiker. Volgens \textcite{Gelenbe2023} wordt ongeveer 10\% van de elektriciteit wereldwijd besteed aan informaticasystemen. Jaarlijks groeit dit percentage en is er nood aan een manier om deze groei in te krimpen of zelfs te stoppen. Daarom wordt energie-efficiënt werken steeds meer noodzakelijk in de software omgeving.  

\subsection{Casus: Davo Group NV}
Omwille dat Davo Group NV een bedrijf is dat duurzaam te werk gaat, willen zij te weten komen hoe ze energie-efficiënt te werk kunnen gaan binnen hun applicatieontwikkeling. Daarvoor willen ze onderzoeken of een verandering van architectuur een invloed heeft op hun verbruik. Het onderzoek zal zich richten op het testen en vergelijken van ontwikkelarchitecturen, waarvan de monolithische architectuur er één van is, op basis van hun energie-efficiëntie, met oog houdend op performantie van de software. Dit is niet alleen interessant voor Davo Group NV, maar ook voor bedrijven die zich inzetten om duurzaam om te gaan in de ICT-wereld. Dit onderzoek kan zo een basis vormen voor de oplossing van een wereldwijd probleem.


\subsection{Onderzoeksvraag en doelstellingen}
De onderzoeksvraag van deze paper is nagaan hoe Davo Group NV energie-efficiënter kan worden op vlak van applicaties door gebruik te maken van een bepaalde ontwikkelarchitectuur, en zo ja, welke.

Volgende doelstellingen dragen bij om een concreet antwoord te voorzien op de onderzoeksvraag:

\begin{itemize}
    \item Literatuurstudie: Gekende ontwikkelarchitecturen in kaart brengen;
    \item POC: Een applicatie uitgewerkt in verschillende ontwikkelarchitecturen;
    \item Resultatenanalyse: De resultaten van de PoC analyseren en hiervan een conclusie vormen.
\end{itemize}

    
%
%Waarover zal je bachelorproef gaan? Introduceer het thema en zorg dat volgende zaken zeker duidelijk aanwezig zijn:
%
%\begin{itemize}
%  \item kaderen thema
%  \item de doelgroep
%  \item de probleemstelling en (centrale) onderzoeksvraag
%  \item de onderzoeksdoelstelling
%\end{itemize}
%
%Denk er aan: een typische bachelorproef is \textit{toegepast onderzoek}, wat betekent dat je start vanuit een concrete probleemsituatie in bedrijfscontext, een \textbf{casus}. Het is belangrijk om je onderwerp goed af te bakenen: je gaat voor die \textit{ene specifieke probleemsituatie} op zoek naar een goede oplossing, op basis van de huidige kennis in het vakgebied.
%
%De doelgroep moet ook concreet en duidelijk zijn, dus geen algemene of vaag gedefinieerde groepen zoals \emph{bedrijven}, \emph{developers}, \emph{Vlamingen}, enz. Je richt je in elk geval op it-professionals, een bachelorproef is geen populariserende tekst. Eén specifiek bedrijf (die te maken hebben met een concrete probleemsituatie) is dus beter dan \emph{bedrijven} in het algemeen.
%
%Formuleer duidelijk de onderzoeksvraag! De begeleiders lezen nog steeds te veel voorstellen waarin we geen onderzoeksvraag terugvinden.
%
%Schrijf ook iets over de doelstelling. Wat zie je als het concrete eindresultaat van je onderzoek, naast de uitgeschreven scriptie? Is het een proof-of-concept, een rapport met aanbevelingen, \ldots Met welk eindresultaat kan je je bachelorproef als een succes beschouwen?

%---------- Stand van zaken ---------------------------------------------------

\section{State-of-the-art}%
\label{sec:state-of-the-art}
\subsection{Ontwikkelarchitecturen}
Ontwikkelarchitecturen zijn een manier om herhalende problemen binnen software development aan te pakken door gebruik te maken van een fundamentele structuur \autocite{Dhaduk2020}. Elk van die architecturen heeft zijn voor en nadelen waardoor het belangrijk is om deze goed te vergelijken of ze in aanmerking komen voor het project. Enkele voorbeelden van ontwikkelarchitecturen zijn: monolieten, gelaagde architectuur, microservices, event-driven architectuur, ...


%todo Extra info ontwikkelarchitecturen

\subsection{Impact software design op energie verbruik}
%todo hoe software related is, zie DOI  10.1145/1235
Energie verbruik (E) is de hoeveelheid stroom (P) dat je verbruikt over een bepaalde tijdsperiode (t), uitgedrukt als \(E = P * t\) met E in Joule, P in Watt en t in seconden. Volgens \textcite{GustavoPinto2017} moeten volgende voorwaarden voldaan zijn om het verbruik te meten van een software applicatie:
\begin{itemize}
    \item het moet actief zijn;
    \item het moet draaien op een hardware platform;
    \item een bepaald scenario neemt plaats;
    \item het wordt uitgevoerd in een bepaalde tijdspanne.
\end{itemize}

Resources zoals CPU, RAM, netwerk, ... verbruiken allemaal stroom en dragen dus bij aan het energie verbruik. Eerdere studies tonen echter aan dat de manier waarop software ontwikkelt wordt het gebruik van die resources kan beïnvloeden. Zo zou het implementeren van design patterns en bepaalde datastructuren, in sommige gevallen, leiden tot een besparing in verbruik van energie \autocite{Georgiou2019}.


\subsection{Performantie}
Performantie is een breed concept. Het is onder andere de snelheid waarop een applicatie reageert op een bepaalde gebruikers input \autocite{Miskolczy2023} en is afhankelijk van verschillende factoren, zoals de complexiteit of het design van de applicatie, het netwerk, de infrastructuur, ... \autocite{Allen2024}. Het meten van performantie gebeurt aan de hand van monitoring. Dit gaat allerlei data verzamelen van de applicatie, waarvan onder andere het CPU- en geheugengebruik, aantal netwerkverzoeken per minuut, vertraging tussen gebruikersinteractie, gemiddelde tijd dat de server nodig heeft om te antwoorden, ... \autocite{Kanjilal2022}. 

De verzamelde data heeft wel degelijk een betekenis. Bij een laag CPU- en geheugengebruik is de applicatie niet alleen meer performant, maar heeft ook een betere kost-efficiëntie. Het kan namelijk snel reageren op gebruikersinvoer, terwijl resources optimaal gebruikt worden.
De capaciteit van aantal netwerkverzoeken per minuut is een goede indicator hoeveel gebruikers dezelfde applicatie/service kunnen gebruiken op hetzelfde moment. De vertraging tussen gebruikersinteractie en de gemiddelde tijd dat de server nodig heeft om te antwoorden gaan hand in hand. Vaak is er na interactie van een gebruiker data nodig van de server, en zal de vertraging die optreedt vaak de tijd zijn die nodig is om de server te bereiken om data op te halen. 

\subsection{Energie-efficiëntie}
Energie-efficiëntie ligt nauw samen met performantie. De resources worden op eenzelfde manier gemonitord, echter wordt andere data verzameld. Het gaat hier niet om in welke mate de resource gebruikt wordt, maar eerder hoeveel energie ze verbruiken \autocite{Kor2015}. Het meten van energie-efficiëntie is niet een eenmalige taak, maar gebeurt best meerdere keren over een bepaald scenario. Zo kan er een beter beeld gevormd worden over het effectieve energie verbruik van een applicatie.


%Hier beschrijf je de \emph{state-of-the-art} rondom je gekozen onderzoeksdomein, d.w.z.\ een inleidende, doorlopende tekst over het onderzoeksdomein van je bachelorproef. Je steunt daarbij heel sterk op de professionele \emph{vakliteratuur}, en niet zozeer op populariserende teksten voor een breed publiek. Wat is de huidige stand van zaken in dit domein, en wat zijn nog eventuele open vragen (die misschien de aanleiding waren tot je onderzoeksvraag!)?
%
%Je mag de titel van deze sectie ook aanpassen (literatuurstudie, stand van zaken, enz.). Zijn er al gelijkaardige onderzoeken gevoerd? Wat concluderen ze? Wat is het verschil met jouw onderzoek?
%
%Verwijs bij elke introductie van een term of bewering over het domein naar de vakliteratuur, bijvoorbeeld~\autocite{Hykes2013}! Denk zeker goed na welke werken je refereert en waarom.
%
%Draag zorg voor correcte literatuurverwijzingen! Een bronvermelding hoort thuis \emph{binnen} de zin waar je je op die bron baseert, dus niet er buiten! Maak meteen een verwijzing als je gebruik maakt van een bron. Doe dit dus \emph{niet} aan het einde van een lange paragraaf. Baseer nooit teveel aansluitende tekst op eenzelfde bron.
%
%Als je informatie over bronnen verzamelt in JabRef, zorg er dan voor dat alle nodige info aanwezig is om de bron terug te vinden (zoals uitvoerig besproken in de lessen Research Methods).

% Voor literatuurverwijzingen zijn er twee belangrijke commando's:
% \dfrac{\autocite{KEY} => (Auteur, jaartal) Gebruik dit als de naam van de auteur
%   geen onderdeel is van de zin.}{_{}}
% \textcite{KEY} => Auteur (jaartal)  Gebruik dit als de auteursnaam wel een
%   functie heeft in de zin (bv. ``Uit onderzoek door Doll & Hill (1954) bleek
%   ...'')

%Je mag deze sectie nog verder onderverdelen in subsecties als dit de structuur van de tekst kan verduidelijken.

%---------- Methodologie ------------------------------------------------------
%todo recheck nieuw onderwerp
\section{Methodologie}%
\label{sec:methodologie}
\subsection{Literatuurstudie}
Als eerste wordt er een uitgebreide literatuurstudie uitgevoerd om de meest gekende software architecturen in kaart te brengen. Hiervan worden best practices, voor- en nadelen alsook performantie aspecten opgezocht. Ook worden tools gezocht om energie-efficiëntie bij software te meten. Om grondige literatuurstudie mogelijk te maken wordt deze fase gespreid over 3 weken, 21 dagen.

\subsection{Requirement analyse en long list}
%todo pas aan
De tweede stap is het uitvoeren van een requirement analyse. Er wordt overlegd met belanghebbenden binnen Davo Group NV welke technische vereisten er aanwezig moeten zijn, zoals programmeertaal voor de PoC, welke tools er gebruikt moeten worden, .... De volgende fase richt zich op het vergelijken van monitoring tools. Elke mogelijkheid wordt dan aan de hand van de beschikbare informatie uit de literatuurstudie vergeleken met de requirements of deze al dan niet de nodige data kunnen capteren. Deze stap wordt uitgewerkt in een tijdspanne van 2 weken, 14 dagen.
\subsection{Short list en PoC}
Als deze lijst met alle mogelijke architecturen is opgesteld, worden er, naast de huidige architectuur die Davo Group NV hanteert voor hun applicaties, nog 2 architecturen gekozen alsook  1 monitoring tool. Dit kan aanschouwt worden als de short list. Als proof of concept wordt voor elke architectuur een vooraf gedefinieerd scenario uitgewerkt waar enkele services die ook aanwezig zijn op de huidige applicaties van Davo Group NV. 

Tijdens het uitwerken van de PoC worden de ondervindingen zoals energieverbruik en performantie gedocumenteerd. De volledige uitwerking van de short list en PoC duurt ongeveer 7 weken. Dit zou genoeg tijd moeten geven om de services te ontwikkelen en uit te rollen op de 2 gekozen microservice architecturen en deze te testen aan de hand van de opgestelde requirements.
\subsection{Vormen conclusie}
In de laatste stap wordt de informatie uit de literatuurstudie in combinatie met de resultaten van de proof of concept geanalyseerd. Zo kan er een conclusie gevormd worden of de keuze van software architectuur voor applicaties een impact heeft op energie-efficiëntie en welke hiervoor het meest optimaal is.

%Hier beschrijf je hoe je van plan bent het onderzoek te voeren. Welke onderzoekstechniek ga je toepassen om elk van je onderzoeksvragen te beantwoorden? Gebruik je hiervoor literatuurstudie, interviews met belanghebbenden (bv.~voor requirements-analyse), experimenten, simulaties, vergelijkende studie, risico-analyse, PoC, \ldots?
%
%Valt je onderwerp onder één van de typische soorten bachelorproeven die besproken zijn in de lessen Research Methods (bv.\ vergelijkende studie of risico-analyse)? Zorg er dan ook voor dat we duidelijk de verschillende stappen terug vinden die we verwachten in dit soort onderzoek!
%
%Vermijd onderzoekstechnieken die geen objectieve, meetbare resultaten kunnen opleveren. Enquêtes, bijvoorbeeld, zijn voor een bachelorproef informatica meestal \textbf{niet geschikt}. De antwoorden zijn eerder meningen dan feiten en in de praktijk blijkt het ook bijzonder moeilijk om voldoende respondenten te vinden. Studenten die een enquête willen voeren, hebben meestal ook geen goede definitie van de populatie, waardoor ook niet kan aangetoond worden dat eventuele resultaten representatief zijn.
%
%Uit dit onderdeel moet duidelijk naar voor komen dat je bachelorproef ook technisch voldoen\-de diepgang zal bevatten. Het zou niet kloppen als een bachelorproef informatica ook door bv.\ een student marketing zou kunnen uitgevoerd worden.
%
%Je beschrijft ook al welke tools (hardware, software, diensten, \ldots) je denkt hiervoor te gebruiken of te ontwikkelen.
%
%Probeer ook een tijdschatting te maken. Hoe lang zal je met elke fase van je onderzoek bezig zijn en wat zijn de concrete \emph{deliverables} in elke fase?

%---------- Verwachte resultaten ----------------------------------------------

\section{Verwacht resultaat, conclusie}%
\label{sec:verwachte_resultaten}
\subsection{Verwachte resultaat}
De literatuurstudie zal leiden tot een duidelijk zicht op bestaande ontwikkelarchitecturen. Het bevat best practices, voor- en nadelen, performantie en indien mogelijk ook eerdere resultaten in verband met energie-efficiëntie.


Bij de requirement analyse is er een overzicht van de technische vereisten die Davo Group NV stelt. Deze zijn de basis voor een lijst van monitoringtools die gebruikt kunnen worden tijdens de uitvoering van de PoC.


Eenmaal de selectie gemaakt is welke ontwikkelarchitecturen gebruikt zullen worden, wordt de proof of concept opgesteld aan de hand van een vooraf gedefinieerd scenario waarin enkele functionaliteiten van hun bestaande applicaties nagebootst worden. De resultaten die hieruit voortvloeien geven inzicht in de energie-efficiëntie van de architectuur.

Het verwachte resultaat is dat er een verschil zal zijn in energie-efficiëntie. Dit is gebaseerd op voorgaande onderzoeken die aantonen dat bij verschillende design patterns en datastructuren ook een verschil aanwezig is en dus logischerwijs hetzelfde zal voorvallen voor de architectuur van de software.

\subsection{Conclusie}
Aan de hand van deze resultaten kan een conclusie gevormd worden. De PoC zal de belangrijkste factor zijn om te beslissen of een keuze van software ontwikkelarchitectuur degelijk impact heeft op energie-efficiëntie, of dus duurzaamheid.

Uit dit onderzoek kunnen we verwachten dat er een eerste beeld gevormd wordt rond energie-efficiëntie in relatie met software ontwikkelarchitecturen, maar het geeft ook een aanzet tot het onderzoeken van meerdere ontwikkelarchitecturen en het optimaliseren ervan met focus op de energie-efficiëntie.
%Hier beschrijf je welke resultaten je verwacht. Als je metingen en simulaties uitvoert, kan je hier al mock-ups maken van de grafieken samen met de verwachte conclusies. Benoem zeker al je assen en de onderdelen van de grafiek die je gaat gebruiken. Dit zorgt ervoor dat je concreet weet welk soort data je moet verzamelen en hoe je die moet meten.
%
%Wat heeft de doelgroep van je onderzoek aan het resultaat? Op welke manier zorgt jouw bachelorproef voor een meerwaarde?
%
%Hier beschrijf je wat je verwacht uit je onderzoek, met de motivatie waarom. Het is \textbf{niet} erg indien uit je onderzoek andere resultaten en conclusies vloeien dan dat je hier beschrijft: het is dan juist interessant om te onderzoeken waarom jouw hypothesen niet overeenkomen met de resultaten.

