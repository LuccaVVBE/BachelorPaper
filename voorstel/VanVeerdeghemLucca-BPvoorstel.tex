
% Gebaseerd op document class `hogent-article'
% zie <https://github.com/HoGentTIN/latex-hogent-article>


\documentclass{hogent-article}

% Invoegen bibliografiebestand
\addbibresource{voorstel.bib}

% Informatie over de opleiding, het vak en soort opdracht
\studyprogramme{Professionele bachelor toegepaste informatica}
\course{Bachelorproef}
\assignmenttype{Onderzoeksvoorstel}


\academicyear{2023-2024} 

% TODO: Werktitel
\title{IT-Optimalisatie: Microservices Architectuur in de Praktijk}

%wat is een efficienter alternatief, meerdere mogelijkheden & 
%vergelijkende literatuurstudie. PoC van de mogelijkheden
\author{Lucca Van Veerdeghem}
\email{lucca.vanveerdeghem@student.hogent.be}


% TODO: Geef de co-promotor op
%Iemand van team progress in DAVO GROUP
\supervisor[Co-promotor]{T. De Geyter (Davo Group NV, \href{mailto:tim.degeyter@davo.be}{tim.degeyter@davo.be})}

% Binnen welke specialisatierichting uit 3TI situeert dit onderzoek zich?
% Kies uit deze lijst:
%
% - Mobile \& Enterprise development
% - AI \& Data Engineering
% - Functional \& Business Analysis
% - System \& Network Administrator
% - Mainframe Expert
% - Als het onderzoek niet past binnen een van deze domeinen specifieer je deze
%   zelf
%
\specialisation{Mobile \& Enterprise development}
\keywords{Microservices, Development, Performance, Container Orchestration}

\begin{document}

\begin{abstract}
Dit onderzoek brengt mogelijkheden in kaart die de bestaande IT-infrastructuur van Davo Group NV kan optimaliseren door over te schakelen naar een microservice-architectuur voor hun klantenportaal. De centrale onderzoeksvraag richt zich op het identificeren van het meest geschikte microservices platform op basis van ontwikkelervaring, functionaliteit en schaalbaarheid. Voor het bekomen van het resultaat wordt er een uitgebreide literatuurstudie, een requirement analyse, een long list, een proof of concept en uiteindelijk een conclusie op basis van de literatuurstudie en PoC uitgewerkt. Het verwachte eindresultaat is een gedetailleerde analyse van microservices, best practices en container orchestration, maar ook inzichten in flexibiliteit, ontwikkeltijd en schaalbaarheid van de gekozen platformen in de PoC. De eindconclusie zal Davo Group NV een leidraad geven in het nemen van beslissingen over hun IT-infrastructuur. Desondanks kan deze bachelorproef ook waardevolle inzichten bieden voor organisaties met een gelijkaardige probleemstelling die overwegen de overstap naar microservices te maken.
    
%  Hier schrijf je de samenvatting van je voorstel, als een doorlopende tekst van één paragraaf. Let op: dit is geen inleiding, maar een samenvattende tekst van heel je voorstel met inleiding (voorstelling, kaderen thema), probleemstelling en centrale onderzoeksvraag, onderzoeksdoelstelling (wat zie je als het concrete resultaat van je bachelorproef?), voorgestelde methodologie, verwachte resultaten en meerwaarde van dit onderzoek (wat heeft de doelgroep aan het resultaat?).
\end{abstract}

\tableofcontents

% De hoofdtekst van het voorstel zit in een apart bestand, zodat het makkelijk
% kan opgenomen worden in de bijlagen van de bachelorproef zelf.
%---------- Inleiding ---------------------------------------------------------

\section{Introductie}%
\label{sec:introductie}
Deze bachelorproef onderzoekt de mogelijke voordelen en uitdagingen in het migreren van een bestaande PHP-webapplicatie naar het ASP.NET framework. Davo Group NV heeft een bestaande web applicatie gebouwd in PHP, maar stellen zich de vraag of een migratie naar ASP.NET een effectievere server-side oplossing zou zijn. De doelgroep van deze bachelorproef zijn de ontwikkelaars binnen Davo Group NV die verantwoordelijk zijn voor de evolutie van de bestaande PHP-webapplicatie.

Het thema gaat concreet over de keuze van server-side frameworks, waarbij er gefocust wordt op het verschil tussen PHP en ASP.NET. Het probleem is namelijk dat er een optimaal server-side framework gevonden moet worden om de huidige PHP-applicatie mogelijks te vervangen. We richten ons specifiek op de technische en prestatiegerelateerde aspecten om de impact op performantie, security en ontwikkeltijd in kaart te brengen. 

De vraag die heerst in deze bachelorproef is of een migratie van de PHP-webapplicatie naar ASP.NET een positieve impact zal hebben.
De bedoeling is om 2 resultaten te verkrijgen. Het eerste resultaat is een grondige analyse van de technische aspecten van PHP en ASP.NET. Het andere resultaat is een aanbeveling of de migratie naar ASP.NET een effectievere oplossing is dan de huidige PHP-webapplicatie.


%
%Waarover zal je bachelorproef gaan? Introduceer het thema en zorg dat volgende zaken zeker duidelijk aanwezig zijn:
%
%\begin{itemize}
%  \item kaderen thema
%  \item de doelgroep
%  \item de probleemstelling en (centrale) onderzoeksvraag
%  \item de onderzoeksdoelstelling
%\end{itemize}
%
%Denk er aan: een typische bachelorproef is \textit{toegepast onderzoek}, wat betekent dat je start vanuit een concrete probleemsituatie in bedrijfscontext, een \textbf{casus}. Het is belangrijk om je onderwerp goed af te bakenen: je gaat voor die \textit{ene specifieke probleemsituatie} op zoek naar een goede oplossing, op basis van de huidige kennis in het vakgebied.
%
%De doelgroep moet ook concreet en duidelijk zijn, dus geen algemene of vaag gedefinieerde groepen zoals \emph{bedrijven}, \emph{developers}, \emph{Vlamingen}, enz. Je richt je in elk geval op it-professionals, een bachelorproef is geen populariserende tekst. Eén specifiek bedrijf (die te maken hebben met een concrete probleemsituatie) is dus beter dan \emph{bedrijven} in het algemeen.
%
%Formuleer duidelijk de onderzoeksvraag! De begeleiders lezen nog steeds te veel voorstellen waarin we geen onderzoeksvraag terugvinden.
%
%Schrijf ook iets over de doelstelling. Wat zie je als het concrete eindresultaat van je onderzoek, naast de uitgeschreven scriptie? Is het een proof-of-concept, een rapport met aanbevelingen, \ldots Met welk eindresultaat kan je je bachelorproef als een succes beschouwen?

%---------- Stand van zaken ---------------------------------------------------

\section{State-of-the-art}%
\label{sec:state-of-the-art}
\subsection{Wat is php?}
\subsubsection{Definitie}
\subsubsection{Voor- en nadelen}
\subsection{Wat is ASP.NET?}
\subsubsection{Definitie}
\subsubsection{Voor- en nadelen}


%Hier beschrijf je de \emph{state-of-the-art} rondom je gekozen onderzoeksdomein, d.w.z.\ een inleidende, doorlopende tekst over het onderzoeksdomein van je bachelorproef. Je steunt daarbij heel sterk op de professionele \emph{vakliteratuur}, en niet zozeer op populariserende teksten voor een breed publiek. Wat is de huidige stand van zaken in dit domein, en wat zijn nog eventuele open vragen (die misschien de aanleiding waren tot je onderzoeksvraag!)?
%
%Je mag de titel van deze sectie ook aanpassen (literatuurstudie, stand van zaken, enz.). Zijn er al gelijkaardige onderzoeken gevoerd? Wat concluderen ze? Wat is het verschil met jouw onderzoek?
%
%Verwijs bij elke introductie van een term of bewering over het domein naar de vakliteratuur, bijvoorbeeld~\autocite{Hykes2013}! Denk zeker goed na welke werken je refereert en waarom.
%
%Draag zorg voor correcte literatuurverwijzingen! Een bronvermelding hoort thuis \emph{binnen} de zin waar je je op die bron baseert, dus niet er buiten! Maak meteen een verwijzing als je gebruik maakt van een bron. Doe dit dus \emph{niet} aan het einde van een lange paragraaf. Baseer nooit teveel aansluitende tekst op eenzelfde bron.
%
%Als je informatie over bronnen verzamelt in JabRef, zorg er dan voor dat alle nodige info aanwezig is om de bron terug te vinden (zoals uitvoerig besproken in de lessen Research Methods).

% Voor literatuurverwijzingen zijn er twee belangrijke commando's:
% \autocite{KEY} => (Auteur, jaartal) Gebruik dit als de naam van de auteur
%   geen onderdeel is van de zin.
% \textcite{KEY} => Auteur (jaartal)  Gebruik dit als de auteursnaam wel een
%   functie heeft in de zin (bv. ``Uit onderzoek door Doll & Hill (1954) bleek
%   ...'')

%Je mag deze sectie nog verder onderverdelen in subsecties als dit de structuur van de tekst kan verduidelijken.

%---------- Methodologie ------------------------------------------------------
\section{Methodologie}%
\label{sec:methodologie}

Hier beschrijf je hoe je van plan bent het onderzoek te voeren. Welke onderzoekstechniek ga je toepassen om elk van je onderzoeksvragen te beantwoorden? Gebruik je hiervoor literatuurstudie, interviews met belanghebbenden (bv.~voor requirements-analyse), experimenten, simulaties, vergelijkende studie, risico-analyse, PoC, \ldots?

Valt je onderwerp onder één van de typische soorten bachelorproeven die besproken zijn in de lessen Research Methods (bv.\ vergelijkende studie of risico-analyse)? Zorg er dan ook voor dat we duidelijk de verschillende stappen terug vinden die we verwachten in dit soort onderzoek!

Vermijd onderzoekstechnieken die geen objectieve, meetbare resultaten kunnen opleveren. Enquêtes, bijvoorbeeld, zijn voor een bachelorproef informatica meestal \textbf{niet geschikt}. De antwoorden zijn eerder meningen dan feiten en in de praktijk blijkt het ook bijzonder moeilijk om voldoende respondenten te vinden. Studenten die een enquête willen voeren, hebben meestal ook geen goede definitie van de populatie, waardoor ook niet kan aangetoond worden dat eventuele resultaten representatief zijn.

Uit dit onderdeel moet duidelijk naar voor komen dat je bachelorproef ook technisch voldoen\-de diepgang zal bevatten. Het zou niet kloppen als een bachelorproef informatica ook door bv.\ een student marketing zou kunnen uitgevoerd worden.

Je beschrijft ook al welke tools (hardware, software, diensten, \ldots) je denkt hiervoor te gebruiken of te ontwikkelen.

Probeer ook een tijdschatting te maken. Hoe lang zal je met elke fase van je onderzoek bezig zijn en wat zijn de concrete \emph{deliverables} in elke fase?

%---------- Verwachte resultaten ----------------------------------------------
\section{Verwacht resultaat, conclusie}%
\label{sec:verwachte_resultaten}

Hier beschrijf je welke resultaten je verwacht. Als je metingen en simulaties uitvoert, kan je hier al mock-ups maken van de grafieken samen met de verwachte conclusies. Benoem zeker al je assen en de onderdelen van de grafiek die je gaat gebruiken. Dit zorgt ervoor dat je concreet weet welk soort data je moet verzamelen en hoe je die moet meten.

Wat heeft de doelgroep van je onderzoek aan het resultaat? Op welke manier zorgt jouw bachelorproef voor een meerwaarde?

Hier beschrijf je wat je verwacht uit je onderzoek, met de motivatie waarom. Het is \textbf{niet} erg indien uit je onderzoek andere resultaten en conclusies vloeien dan dat je hier beschrijft: het is dan juist interessant om te onderzoeken waarom jouw hypothesen niet overeenkomen met de resultaten.



\printbibliography[heading=bibintoc]

\end{document}