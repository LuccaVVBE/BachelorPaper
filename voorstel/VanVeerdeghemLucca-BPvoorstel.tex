
% Gebaseerd op document class `hogent-article'
% zie <https://github.com/HoGentTIN/latex-hogent-article>


\documentclass{hogent-article}

% Invoegen bibliografiebestand
\addbibresource{voorstel.bib}

% Informatie over de opleiding, het vak en soort opdracht
\studyprogramme{Professionele bachelor toegepaste informatica}
\course{Bachelorproef}
\assignmenttype{Onderzoeksvoorstel}


\academicyear{2023-2024} 

% TODO: Werktitel
\title{IT-Optimalisatie: Microservices Architectuur in de Praktijk}

%wat is een efficienter alternatief, meerdere mogelijkheden & 
%vergelijkende literatuurstudie. PoC van de mogelijkheden
\author{Lucca Van Veerdeghem}
\email{lucca.vanveerdeghem@student.hogent.be}


% TODO: Geef de co-promotor op
%Iemand van team progress in DAVO GROUP
\supervisor[Co-promotor]{T. De Geyter (Davo Group NV, \href{mailto:tim.degeyter@davo.be}{tim.degeyter@davo.be})}

% Binnen welke specialisatierichting uit 3TI situeert dit onderzoek zich?
% Kies uit deze lijst:
%
% - Mobile \& Enterprise development
% - AI \& Data Engineering
% - Functional \& Business Analysis
% - System \& Network Administrator
% - Mainframe Expert
% - Als het onderzoek niet past binnen een van deze domeinen specifieer je deze
%   zelf
%
\specialisation{Mobile \& Enterprise development}
\keywords{Microservices, Development, Performance, Container Orchestration}

\begin{document}

\begin{abstract}
Dit onderzoek brengt mogelijkheden in kaart die de bestaande IT-infrastructuur van Davo Group NV kan optimaliseren door over te schakelen naar een microservice-architectuur voor hun klantenportaal. De centrale onderzoeksvraag richt zich op het identificeren van het meest geschikte microservices platform op basis van ontwikkelervaring, functionaliteit en schaalbaarheid. Voor het bekomen van het resultaat wordt er een uitgebreide literatuurstudie, een requirement analyse, een long list, een proof of concept en uiteindelijk een conclusie op basis van de literatuurstudie en PoC uitgewerkt. Het verwachte eindresultaat is een gedetailleerde analyse van microservices, best practices en container orchestration, maar ook inzichten in flexibiliteit, ontwikkeltijd en schaalbaarheid van de gekozen platformen in de PoC. De eindconclusie zal Davo Group NV een leidraad geven in het nemen van beslissingen over hun IT-infrastructuur. Desondanks kan deze bachelorproef ook waardevolle inzichten bieden voor organisaties met een gelijkaardige probleemstelling die overwegen de overstap naar microservices te maken.
    
%  Hier schrijf je de samenvatting van je voorstel, als een doorlopende tekst van één paragraaf. Let op: dit is geen inleiding, maar een samenvattende tekst van heel je voorstel met inleiding (voorstelling, kaderen thema), probleemstelling en centrale onderzoeksvraag, onderzoeksdoelstelling (wat zie je als het concrete resultaat van je bachelorproef?), voorgestelde methodologie, verwachte resultaten en meerwaarde van dit onderzoek (wat heeft de doelgroep aan het resultaat?).
\end{abstract}

\tableofcontents

% De hoofdtekst van het voorstel zit in een apart bestand, zodat het makkelijk
% kan opgenomen worden in de bijlagen van de bachelorproef zelf.
%---------- Inleiding ---------------------------------------------------------

\section{Introductie}%
\label{sec:introductie}
\subsection{Microservices in de bedrijfswereld}
Microservices groeien de laatste jaren in populariteit in de bedrijfswereld. Organisaties gebruiken deze architectuur meer en meer voor het ontwikkelen en beheren van moderne applicaties. Ze brengen aanzienlijke verbeteringen mee zoals flexibiliteit, onderhoudbaarheid en schaalbaarheid in vergelijking met de traditionele manier van ontwikkeling.

Veel sectoren schakelen naar microservices om meer agile en schaalbare oplossingen te realiseren. Het laat toe om grote complexe applicaties op te bouwen als meerdere onafhankelijke services die afzonderlijk ontwikkeld, geïmplementeerd en geschaald kunnen worden. Dit maakt applicaties beter beheersbaar en sneller aanpasbaar aan veranderende vereisten.

\subsection{Casus: Davo Group NV}
Gezien de groeiende populariteit van microservices, overweegt ook Davo Group NV om over te stappen naar een microservices-architectuur voor hun klantenportaal, onder andere door de voordelen die de architectuur met zich meebrengt. Het onderzoek dat uitgevoerd zal worden richt zich op de potentiële optimalisatie van de IT-infrastructuur bij Davo Group NV, maar kan ook gebruikt worden door bedrijven met een vergelijkbare probleemstelling. Het huidige klantenportaal bestaat uit meerdere functionaliteiten, zoals inloggen, aanvragen van producten \& services, opvolgen van dataverbruik, ...

Door de implementatie van microservices kan het systeem modulair opgebouwd worden waardoor flexibiliteit verhoogd wordt en ontwikkeltijd verkort. Het biedt de mogelijkheid om afzonderlijk services te gaan schalen, zoals vermeld in het vorige sub-hoofdstuk.

\subsection{Onderzoeksvraag en doelstellingen}
De onderzoeksvraag van deze paper is hoe Davo Group NV de bestaande IT-infrastuctuur kan optimaliseren door naar een microservices-\newline architectuur over te stappen, en welk platform is het meest geschikt volgens ontwikkelervaring, functionaliteit en schaalbaarheid.

Volgende doelstellingen dragen bij om een concreet antwoord te voorzien op de onderzoeksvraag:

\begin{itemize}
    \item Analyse: Mogelijkheden en uitdagingen van microservices in kaart brengen
    \item POC: Enkele services uitgewerkt in 2\newline microservices-architecturen 
    \item Aanbevelingen: Op basis van analyse en ondervindingen PoC mogelijkheden bespreken
\end{itemize}

    
%
%Waarover zal je bachelorproef gaan? Introduceer het thema en zorg dat volgende zaken zeker duidelijk aanwezig zijn:
%
%\begin{itemize}
%  \item kaderen thema
%  \item de doelgroep
%  \item de probleemstelling en (centrale) onderzoeksvraag
%  \item de onderzoeksdoelstelling
%\end{itemize}
%
%Denk er aan: een typische bachelorproef is \textit{toegepast onderzoek}, wat betekent dat je start vanuit een concrete probleemsituatie in bedrijfscontext, een \textbf{casus}. Het is belangrijk om je onderwerp goed af te bakenen: je gaat voor die \textit{ene specifieke probleemsituatie} op zoek naar een goede oplossing, op basis van de huidige kennis in het vakgebied.
%
%De doelgroep moet ook concreet en duidelijk zijn, dus geen algemene of vaag gedefinieerde groepen zoals \emph{bedrijven}, \emph{developers}, \emph{Vlamingen}, enz. Je richt je in elk geval op it-professionals, een bachelorproef is geen populariserende tekst. Eén specifiek bedrijf (die te maken hebben met een concrete probleemsituatie) is dus beter dan \emph{bedrijven} in het algemeen.
%
%Formuleer duidelijk de onderzoeksvraag! De begeleiders lezen nog steeds te veel voorstellen waarin we geen onderzoeksvraag terugvinden.
%
%Schrijf ook iets over de doelstelling. Wat zie je als het concrete eindresultaat van je onderzoek, naast de uitgeschreven scriptie? Is het een proof-of-concept, een rapport met aanbevelingen, \ldots Met welk eindresultaat kan je je bachelorproef als een succes beschouwen?

%---------- Stand van zaken ---------------------------------------------------

\section{State-of-the-art}%
\label{sec:state-of-the-art}
\subsection{Monolithische applicaties}
Monolithische applicaties is de traditionele manier van softwareontwikkeling. Volgens \textcite{Dragoni2016} is een monolithische applicatie een applicatie waar alle modules samenhangen en niet afzonderlijk uitvoerbaar zijn. We zien dat bij mainstream ontwikkeltalen van server-side applicaties, zoals Java, C en Python er abstracties aanwezig zijn om de complexiteit op te splitsen in modules. Echter zal elke applicatie dat uitgewerkt is in een van deze talen bij compilatie van vanaf 1 executable, zijnde de monoliet, waardoor er enkel gebruik gemaakt wordt van resources op 1 systeem. De modules van de monolithische applicatie zijn dus afhankelijk van de gedeelde resources en niet onafhankelijk uitvoerbaar.

%Extra info monolithische applicatie?
\subsection{Microservice architectuur}
Microservices is een nieuwe trend in de software architectuur waar grote, complexe applicaties verdeeld worden in kleinere services. \textcite{Dragoni2016} definieert een microservice architectuur als een gedistribueerde applicatie waar alle modules microservices zijn. Dit zijn samenhangende maar onafhankelijke processen dat met elkaar communiceren aan de hand van API calls. Ze worden apart gedeployed en krijgen hun eigen resources en database toegewezen. 

\textcite{Blinowski2022} stelt dat de hoofdprincipes van een microservice architectuur bestaan uit: één verantwoordelijkheid per service, autonoom kunnen runnen en afgeschermd zijn achter een API. Dit laat toe, afhankelijk van het gebruik van de service, om beter resources te kunnen verdelen. Aangezien de services autonoom kunnen runnen moeten ze niet op hetzelfde systeem draaien en al zeker niet op het gemeenschappelijke proces zoals gebeurt op een monolithische applicatie\autocite{Dragoni2017}. Hierdoor is het mogelijk dat wanneer één service uitvalt, de applicatie grotendeels blijft werken afhankelijk van de impact ervan. Elke service is toegankelijk via een API die zo hun data kunnen doorgeven aan een andere service wanneer die deze nodig heeft. Dit is dan weer een uitdaging om bij elke API call genoeg informatie mee te geven wanneer verschillende services verschillende data nodig hebben.

\subsection{Container orkestratie}
Containers zijn een ideale manier om microservices in te runnen door hun lage overhead en snelle deploymentmogelijkheden\autocite{Jawarneh2019}. Dit is ideaal wanneer er weinig services zijn omdat het beheer ervan vrij eenvoudig blijft. Eenmaal de applicatie groter wordt en meer microservices ingezet worden is het moeilijk om een overzicht te behouden en is het interessant om een extra laag toe te voegen in de architectuur, namelijk de container orkestratie. Deze zorgt ervoor dat het beheren van elke container zijn resources automatisch kan verlopen, alsook regelt deze de coördinatie en communicatie tussen de containers en dus ook de microservices\autocite{Liu2020}. 
%Hier beschrijf je de \emph{state-of-the-art} rondom je gekozen onderzoeksdomein, d.w.z.\ een inleidende, doorlopende tekst over het onderzoeksdomein van je bachelorproef. Je steunt daarbij heel sterk op de professionele \emph{vakliteratuur}, en niet zozeer op populariserende teksten voor een breed publiek. Wat is de huidige stand van zaken in dit domein, en wat zijn nog eventuele open vragen (die misschien de aanleiding waren tot je onderzoeksvraag!)?
%
%Je mag de titel van deze sectie ook aanpassen (literatuurstudie, stand van zaken, enz.). Zijn er al gelijkaardige onderzoeken gevoerd? Wat concluderen ze? Wat is het verschil met jouw onderzoek?
%
%Verwijs bij elke introductie van een term of bewering over het domein naar de vakliteratuur, bijvoorbeeld~\autocite{Hykes2013}! Denk zeker goed na welke werken je refereert en waarom.
%
%Draag zorg voor correcte literatuurverwijzingen! Een bronvermelding hoort thuis \emph{binnen} de zin waar je je op die bron baseert, dus niet er buiten! Maak meteen een verwijzing als je gebruik maakt van een bron. Doe dit dus \emph{niet} aan het einde van een lange paragraaf. Baseer nooit teveel aansluitende tekst op eenzelfde bron.
%
%Als je informatie over bronnen verzamelt in JabRef, zorg er dan voor dat alle nodige info aanwezig is om de bron terug te vinden (zoals uitvoerig besproken in de lessen Research Methods).

% Voor literatuurverwijzingen zijn er twee belangrijke commando's:
% \dfrac{\autocite{KEY} => (Auteur, jaartal) Gebruik dit als de naam van de auteur
%   geen onderdeel is van de zin.}{_{}}
% \textcite{KEY} => Auteur (jaartal)  Gebruik dit als de auteursnaam wel een
%   functie heeft in de zin (bv. ``Uit onderzoek door Doll & Hill (1954) bleek
%   ...'')

%Je mag deze sectie nog verder onderverdelen in subsecties als dit de structuur van de tekst kan verduidelijken.

%---------- Methodologie ------------------------------------------------------
\section{Methodologie}%
\label{sec:methodologie}
\subsection{Literatuurstudie}
Als eerste wordt er een uitgebreide literatuurstudie uitgevoerd om de mogelijkheden en uitdagingen in kaart te brengen van microservices. Ook wordt er meer informatie over best practices, concepten en containerorkestratie in de microservice-architectuur onderzocht. Om grondige analyse mogelijk te maken wordt deze fase gespreid over 2 weken, 14 dagen.
\subsection{Requirement analyse en long list}
De tweede stap is het uitvoeren van een requirement analyse. Er wordt overlegd met belanghebbenden binnen Davo Group NV welke functionele en niet-functionele vereisten er aanwezig moeten zijn en worden gerangschikt volgens het MoSCoW-principe. Op basis van deze informatie kan er overgegaan worden naar een volgende fase waar verschillende alternatieven, zoals Kubernetes\footnote{https://kubernetes.io/} en Docker Swarm\footnote{https://docs.docker.com/engine/swarm/}, opgelijst worden. Elk alternatief wordt dan aan de hand van beschikbare informatie vergeleken met de requirements of deze al dan niet voldoen. Deze stap wordt uitgewerkt in een tijdspanne van 3 weken, 21 dagen.
\subsection{Short list en PoC}
Als deze lijst met alle mogelijke platformen is opgesteld, worden de 2 meest passende gekozen. Op beide platformen wordt een proof of concept opgesteld. Hiervoor wordt een vooraf gedefinieerd scenario uitgewerkt waar enkele services die ook aanwezig zijn op het klantenportaal van Davo Group NV worden nagebootst. 

Tijdens het uitwerken van de PoC worden de ondervindingen zoals flexibiliteit, ontwikkeltijd en schaalbaarheid in functie van de eerder opgestelde requirements gedocumenteerd. De volledige uitwerking van de short list en PoC duurt ongeveer 7 weken. Dit zou genoeg tijd moeten geven om de services te ontwikkelen en uit te rollen op de 2 gekozen microservice architecturen en deze te testen aan de hand van de opgestelde requirements.
\subsection{Vormen conclusie}
In de laatste stap wordt de literatuurstudie in combinatie met de ondervindingen van de proof of concept geanalyseerd. Zo kan er een conclusie gevormd worden of er al dan niet overgeschakeld kan worden naar een microservice architectuur en welk platform er best gebruikt kan worden. Deze laatste stap neemt de resterende 2 weken in beslag, wat zeker toelaat om de theorie met de praktijk objectief te vergelijken.
%Hier beschrijf je hoe je van plan bent het onderzoek te voeren. Welke onderzoekstechniek ga je toepassen om elk van je onderzoeksvragen te beantwoorden? Gebruik je hiervoor literatuurstudie, interviews met belanghebbenden (bv.~voor requirements-analyse), experimenten, simulaties, vergelijkende studie, risico-analyse, PoC, \ldots?
%
%Valt je onderwerp onder één van de typische soorten bachelorproeven die besproken zijn in de lessen Research Methods (bv.\ vergelijkende studie of risico-analyse)? Zorg er dan ook voor dat we duidelijk de verschillende stappen terug vinden die we verwachten in dit soort onderzoek!
%
%Vermijd onderzoekstechnieken die geen objectieve, meetbare resultaten kunnen opleveren. Enquêtes, bijvoorbeeld, zijn voor een bachelorproef informatica meestal \textbf{niet geschikt}. De antwoorden zijn eerder meningen dan feiten en in de praktijk blijkt het ook bijzonder moeilijk om voldoende respondenten te vinden. Studenten die een enquête willen voeren, hebben meestal ook geen goede definitie van de populatie, waardoor ook niet kan aangetoond worden dat eventuele resultaten representatief zijn.
%
%Uit dit onderdeel moet duidelijk naar voor komen dat je bachelorproef ook technisch voldoen\-de diepgang zal bevatten. Het zou niet kloppen als een bachelorproef informatica ook door bv.\ een student marketing zou kunnen uitgevoerd worden.
%
%Je beschrijft ook al welke tools (hardware, software, diensten, \ldots) je denkt hiervoor te gebruiken of te ontwikkelen.
%
%Probeer ook een tijdschatting te maken. Hoe lang zal je met elke fase van je onderzoek bezig zijn en wat zijn de concrete \emph{deliverables} in elke fase?

%---------- Verwachte resultaten ----------------------------------------------
\section{Verwacht resultaat, conclusie}%
\label{sec:verwachte_resultaten}
\subsection{Verwachte resultaten}
De literatuurstudie zal een antwoord bieden op de mogelijkheden en uitdagingen van microservices. Het resultaat is een gedetailleerde analyse over best practices en container orkestratie, wat de basis is voor het verdere onderzoek.

Bij de requirement analyse is er een overzicht van de functionele en niet-functionele requirements die Davo Group NV stelt voor het klantenportaal. Deze zullen de basis zijn van de long list, een lijst met meerdere opties die aan de requirements kunnen voldoen. 

Op basis van die long list wordt er een proof of concept opgesteld aan de hand van een vooraf gedefinieerd scenario waarin enkele functionaliteiten van het klantenportaal nagebootst worden. De PoC wordt uitgewerkt op de twee meest geschikte microservice platformen. De resultaten die hieruit voortvloeien geven inzicht in de flexibiliteit, ontwikkeltijd en schaalbaarheid van het platform, rekening houdend met de vastgestelde requirements.
\newline
\newline
\subsection{Conclusie}
Aan de hand van deze resultaten kan een conclusie gevormd worden. Als de PoC succesvol is bij minstens één microservice platform, kan geconcludeerd worden dat een overstap naar zo een architectuur voor het klantenportaal haalbaar en waardevol is. 

Zowel de literatuurstudie en de resultaten van de PoC worden samen geanalyseerd om te bepalen welk platform het meest geschikt is voor de behoeften van Davo Group NV. Uit deze analyse kan dan een aanbeveling gedaan worden voor het meest geschikte platform. Dit zal een waardevolle leidraad vormen voor Davo Group NV bij de beslissing over hun IT-infrastructuur en eventuele verdere stappen in het implementatieproces.
%Hier beschrijf je welke resultaten je verwacht. Als je metingen en simulaties uitvoert, kan je hier al mock-ups maken van de grafieken samen met de verwachte conclusies. Benoem zeker al je assen en de onderdelen van de grafiek die je gaat gebruiken. Dit zorgt ervoor dat je concreet weet welk soort data je moet verzamelen en hoe je die moet meten.
%
%Wat heeft de doelgroep van je onderzoek aan het resultaat? Op welke manier zorgt jouw bachelorproef voor een meerwaarde?
%
%Hier beschrijf je wat je verwacht uit je onderzoek, met de motivatie waarom. Het is \textbf{niet} erg indien uit je onderzoek andere resultaten en conclusies vloeien dan dat je hier beschrijft: het is dan juist interessant om te onderzoeken waarom jouw hypothesen niet overeenkomen met de resultaten.



\printbibliography[heading=bibintoc]

\end{document}