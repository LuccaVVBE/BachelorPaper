%%=============================================================================
%% Methodologie
%%=============================================================================

\chapter{\IfLanguageName{dutch}{Methodologie}{Methodology}}%
\label{ch:methodologie}

%% TODO: In dit hoofstuk geef je een korte toelichting over hoe je te werk bent
%% gegaan. Verdeel je onderzoek in grote fasen, en licht in elke fase toe wat
%% de doelstelling was, welke deliverables daar uit gekomen zijn, en welke
%% onderzoeksmethoden je daarbij toegepast hebt. Verantwoord waarom je
%% op deze manier te werk gegaan bent.
%% 
%% Voorbeelden van zulke fasen zijn: literatuurstudie, opstellen van een
%% requirements-analyse, opstellen long-list (bij vergelijkende studie),
%% selectie van geschikte tools (bij vergelijkende studie, "short-list"),
%% opzetten testopstelling/PoC, uitvoeren testen en verzamelen
%% van resultaten, analyse van resultaten, ...
%%
%% !!!!! LET OP !!!!!
%%
%% Het is uitdrukkelijk NIET de bedoeling dat je het grootste deel van de corpus
%% van je bachelorproef in dit hoofstuk verwerkt! Dit hoofdstuk is eerder een
%% kort overzicht van je plan van aanpak.
%%
%% Maak voor elke fase (behalve het literatuuronderzoek) een NIEUW HOOFDSTUK aan
%% en geef het een gepaste titel.
%todo HERSCHRIJVEN!!!!!
\section{Literatuurstudie}
In deze fase is uitgebreid literatuuronderzoek uitgevoerd om inzicht te krijgen in de bestaande kennis en technologieën met betrekking tot het onderwerp van de bachelorproef, zijnde het energieverbruik van ontwikkelarchitecturen. Dit omvatte het bestuderen van relevante wetenschappelijke artikelen, boeken en online bronnen om een stevige basis van begrip op te bouwen.


\section{Short list}
Op basis van de literatuurstudie is een short list samengesteld van potentiële methoden, technologieën of tools die relevant zijn voor het onderzoek en gebruikt kunnen worden in de Proof of Concept. Deze selectie is gebaseerd op criteria zoals relevantie, toepasbaarheid, en beschikbaarheid van middelen.


\section{Opzetten PoC}
Na het vaststellen van de short list is een Proof of Concept (PoC) opgezet. Dit omvatte het ontwerpen en implementeren van een applicatie binnen verschillende architecturen.


\section{Uitvoeren testen op PoC}
Met de PoC opgezet, werden gestructureerde tests uitgevoerd om het energieverbruik per architectuur te beoordelen. Hierbij zijn diverse metingen en observaties gedaan om data te verzamelen.


\section{Analyse van testresultaten}
De verzamelde data uit de tests op de PoC zijn geanalyseerd om inzicht te krijgen in de energie-efficiëntie van de verschillende architecturen. 

\section{Conclusie}
Op basis van de uitgevoerde tests en de analyse van de resultaten worden conclusies getrokken met betrekking tot de geschiktheid en toepasbaarheid van de onderzochte methoden of technologieën. Ook worden eventuele suggesties voor toekomstig onderzoek of verbeteringen aangedragen.
