%%=============================================================================
%% Voorwoord
%%=============================================================================

\chapter*{\IfLanguageName{dutch}{Woord vooraf}{Preface}}%
\label{ch:voorwoord}

%% TODO:
%% Het voorwoord is het enige deel van de bachelorproef waar je vanuit je
%% eigen standpunt (``ik-vorm'') mag schrijven. Je kan hier bv. motiveren
%% waarom jij het onderwerp wil bespreken.
%% Vergeet ook niet te bedanken wie je geholpen/gesteund/... heeft

Met trots presenteer ik hierbij mijn bachelorproef, getiteld "Onderzoek naar vermindering van energieverbruik binnen softwareontwikkeling." Gedurende de afgelopen periode heb ik me verdiept in dit boeiende onderwerp, gedreven door mijn passie voor softwareontwikkeling en de voortdurende zoektocht naar optimalisatie ervan.\\

De keuze voor dit onderwerp kwam voort uit mijn interesse voor de technische aspecten van softwareontwikkeling en in het verbeteren van resource gebruik van software. In een tijd waarin duurzaamheid en energie-efficiëntie steeds belangrijker worden, is het verminderen van het energieverbruik binnen softwareontwikkeling een relevant en urgent vraagstuk.\\

Gedurende mijn onderzoekstraject heb ik het voorrecht gehad om te kunnen rekenen op de waardevolle begeleiding van mijn copromotor, Huzeyfe Kaymak. Ondanks zijn drukke agenda heeft hij tijd vrijgemaakt om mijn vragen te beantwoorden, mij te voorzien van waardevolle bijdragen en mij te begeleiden bij mijn onderzoek. Zijn expertise en toewijding hebben mijn onderzoek verrijkt en bijgedragen aan het succes ervan.\\

Tevens wil ik mijn dank uitspreken aan mijn familie, die mij gedurende dit hele proces heeft gesteund en van waardevolle feedback heeft voorzien. Hun aanmoediging en steun hebben mij gemotiveerd om door te zetten, zelfs tijdens de uitdagende momenten van dit onderzoeksavontuur.\\

Dit onderzoek zou niet mogelijk zijn geweest zonder de steun en inzet van deze mensen, en ik ben hen dan ook enorm dankbaar. Ik hoop dat dit onderzoek een waardevolle bijdrage levert aan het vakgebied van de softwareontwikkeling en bijdraagt aan het streven naar een duurzamere toekomst, alsook een aanstoot geeft tot verder onderzoek.\\

Met vriendelijke groet,

Lucca Van Veerdeghem