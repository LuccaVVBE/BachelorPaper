%%=============================================================================
%% Samenvatting
%%=============================================================================

% TODO: De "abstract" of samenvatting is een kernachtige (~ 1 blz. voor een
% thesis) synthese van het document.
%
% Een goede abstract biedt een kernachtig antwoord op volgende vragen:
%
% 1. Waarover gaat de bachelorproef?
% 2. Waarom heb je er over geschreven?
% 3. Hoe heb je het onderzoek uitgevoerd?
% 4. Wat waren de resultaten? Wat blijkt uit je onderzoek?
% 5. Wat betekenen je resultaten? Wat is de relevantie voor het werkveld?
%
% Daarom bestaat een abstract uit volgende componenten:
%
% - inleiding + kaderen thema
% - probleemstelling
% - (centrale) onderzoeksvraag
% - onderzoeksdoelstelling
% - methodologie
% - resultaten (beperk tot de belangrijkste, relevant voor de onderzoeksvraag)
% - conclusies, aanbevelingen, beperkingen
%
% LET OP! Een samenvatting is GEEN voorwoord!

%%---------- Nederlandse samenvatting -----------------------------------------
%
% TODO: Als je je bachelorproef in het Engels schrijft, moet je eerst een
% Nederlandse samenvatting invoegen. Haal daarvoor onderstaande code uit
% commentaar.
% Wie zijn bachelorproef in het Nederlands schrijft, kan dit negeren, de inhoud
% wordt niet in het document ingevoegd.

\IfLanguageName{english}{%
\selectlanguage{dutch}
\chapter*{Samenvatting}
\lipsum[1-4]
\selectlanguage{english}
}{}

%%---------- Samenvatting -----------------------------------------------------
% De samenvatting in de hoofdtaal van het document

\chapter*{\IfLanguageName{dutch}{Samenvatting}{Abstract}}
Deze bachelorproef onderzoekt het energie verbruik van verschillende softwarearchitecturen voor eenzelfde applicatie. Het doel van dit onderzoek is om inzicht te krijgen in welke architectuur het meest geschikt is voor het verminderen van energieverbruik, wat kan leiden tot kostenbesparingen en een positieve impact op het milieu.\\

Het onderzoek is uitgevoerd door de combinatie van een grondige literatuurstudie, het opstellen van een shortlist van relevante architectuur- en meetmethoden, het ontwikkelen van een Proof of Concept (PoC), het uitvoeren van gestructureerde tests op de PoC, en ten slotte de analyse van de testresultaten.\\

De resultaten tonen aan dat <In te vullen als resultaten bekend zijn>.\\
%de microservices-architectuur over het algemeen energie-efficiënter is dan de monolietarchitectuur voor het klantenportaal van Davo Group. Deze bevindingen zijn relevant voor het werkveld omdat ze aangeven welke architectuur het meest geschikt is voor het verminderen van energieverbruik in gelijkaardige toepassingen. Het onderzoek biedt waardevolle inzichten voor Davo Group en draagt bij aan een groter begrip van energie-efficiëntie in softwareontwikkeling.

De conclusie van dit onderzoek is <In te vullen als conclusie gevormd is>.
%kunnen worden gebruikt om toekomstige beslissingen te informeren over de architectuur van het klantenportaal van Davo Group, en bieden tevens suggesties voor verder onderzoek en mogelijke verbeteringen in energie-efficiëntie in softwareontwikkeling.
