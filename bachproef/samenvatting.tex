%%=============================================================================
%% Samenvatting
%%=============================================================================

% TODO: De "abstract" of samenvatting is een kernachtige (~ 1 blz. voor een
% thesis) synthese van het document.
%
% Een goede abstract biedt een kernachtig antwoord op volgende vragen:
%
% 1. Waarover gaat de bachelorproef?
% 2. Waarom heb je er over geschreven?
% 3. Hoe heb je het onderzoek uitgevoerd?
% 4. Wat waren de resultaten? Wat blijkt uit je onderzoek?
% 5. Wat betekenen je resultaten? Wat is de relevantie voor het werkveld?
%
% Daarom bestaat een abstract uit volgende componenten:
%
% - inleiding + kaderen thema
% - probleemstelling
% - (centrale) onderzoeksvraag
% - onderzoeksdoelstelling
% - methodologie
% - resultaten (beperk tot de belangrijkste, relevant voor de onderzoeksvraag)
% - conclusies, aanbevelingen, beperkingen
%
% LET OP! Een samenvatting is GEEN voorwoord!

%%---------- Nederlandse samenvatting -----------------------------------------
%
% TODO: Als je je bachelorproef in het Engels schrijft, moet je eerst een
% Nederlandse samenvatting invoegen. Haal daarvoor onderstaande code uit
% commentaar.
% Wie zijn bachelorproef in het Nederlands schrijft, kan dit negeren, de inhoud
% wordt niet in het document ingevoegd.

\IfLanguageName{english}{%
\selectlanguage{dutch}
\chapter*{Samenvatting}
\lipsum[1-4]
\selectlanguage{english}
}{}

%%---------- Samenvatting -----------------------------------------------------
% De samenvatting in de hoofdtaal van het document

\chapter*{\IfLanguageName{dutch}{Samenvatting}{Abstract}}
Dit onderzoek richtte zich op het analyseren van de impact van architecturale veranderingen op het energieverbruik van een applicatie. De centrale onderzoeksvraag was: "Wat is de impact op energieverbruik bij een architecturale verandering van een applicatie?".\\

Om deze vraag te beantwoorden, werd een Proof of Concept (PoC) applicatie ontwikkeld in zowel een monolithische als een microservice versie. Beide versies werden gehost in een docker omgeving op een Ubuntu systeem en ondergingen diverse scenario's om het energieverbruik te meten. Specifieke methoden en controllers werden geïmplementeerd om gebruikers te laten inloggen en het verbruik van telefoonnummers te raadplegen, waardoor zowel resource-intensieve als niet resource-intensieve scenario’s werden gesimuleerd.\\

De testresultaten toonden aan dat de microservices aanzienlijk meer energie verbruiken dan monolithische applicaties, met een verschil van ongeveer 75\%. Dit verschil is evenredig aan de langere tijd die nodig is voor het uitvoeren van de scenario’s bij de microservices. Ondanks het hogere totale energieverbruik, hadden de microservices een lager gemiddeld wattverbruik vergeleken met de monolithische applicaties. Bovendien vertoonden monolithische applicaties grotere spreiding en variabiliteit in hun energieverbruik, wat wijst op inconsistente prestaties.\\

De analyse liet verder zien dat bij lage belasting het verschil in energieverbruik minimaal is, maar bij hoge belasting verbruiken microservices aanzienlijk meer tijd en dus ook meer energie, ondanks hun lagere gemiddelde wattverbruik. Dit kan verklaard worden door de afhankelijkheid van de microservices aan API-responstijden, terwijl monolithische applicaties direct kunnen reageren op inkomende verzoeken.\\

Hoewel deze bevindingen relevant zijn voor ontwikkelaars en bedrijven die streven naar een vermindering van energieverbruik bij softwareontwikkeling, zijn er enkele belangrijke opmerkingen. Het onderzoek bevestigde dat een microservice architectuur op het gebied van energie besparen niet altijd de beste optie is, alhoewel ze een lager gemiddeld wattverbruik aanhouden.\\

Toekomstig onderzoek zou zich kunnen richten op andere software ontwikkelarchitecturen, zoals serverless of event-driven architecturen, om een beter inzicht te krijgen in hoe verschillende architecturen het energieverbruik beïnvloeden. Dit zou kunnen bijdragen aan beter geïnformeerde beslissingen over architecturale keuzes en hun implicaties voor energieverbruik.\\

Deze studie biedt waardevolle inzichten en draagt bij aan het vakgebied door de impact van architecturale keuzes op energieverbruik te belichten, wat belangrijk is voor de ontwikkeling van duurzame softwareoplossingen.